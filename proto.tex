
\documentclass[a4paper]{article}
\newfont{\foo}{proto}
\usepackage{graphicx}
\usepackage{textcomp}
\usepackage{tipa}
\usepackage{multirow}

\title{Language}
\author{L\'aszl\'o Luk\'acs}

\begin{document}

\newcommand{\DIG}[1]{\reflectbox{\foo #1}}
\newcommand{\DIGA}{\DIG{A}}
\newcommand{\DIGB}{\DIG{B}}
\newcommand{\DIGC}{\DIG{C}}
\newcommand{\DIGD}{\DIG{D}}
\newcommand{\DIGE}{\DIG{E}}
\newcommand{\DIGF}{\DIG{F}}

\newcommand{\LET}[1]{\rotatebox[origin=c]{-90}{\reflectbox{\foo #1}}}
\newcommand{\VU}{\LET W}
\newcommand{\YOE}{\LET X}
\newcommand{\JEE}{\LET Z}
\newcommand{\NE}{\LET a}
\newcommand{\MOE}{\LET b}
\newcommand{\KI}{\LET c}

\newcommand{\SP}{\LET Y}

\newcommand{\DOWN}[1]{\rotatebox[origin=c]{-90}{#1}}
\newcommand{\UP}[1]{\rotatebox[origin=c]{-90}{\reflectbox{#1}}}

\newcommand{\Q}[1]{\textlangle #1\textrangle}

\maketitle

\tableofcontents

\DOWN{
\DIG0 \DIG1 \DIG2 \DIG3 \DIG4 \DIG5
}
\UP{
\DIG6 \DIG7 \DIG8 \DIG9 \DIGA\ \DIGB\
}
\DOWN{
\DIGC\ \DIGD\ \DIGE\ \DIGF\ \VU\ \YOE
}
\UP{
\VU\ \YOE\ \JEE\ \NE\ \MOE\ \KI\
}
\DOWN{
\KI\ \MOE\ \NE\ \JEE\ \YOE\ \VU
}

The symbol for \Q{nene} is \DOWN{\NE\ \NE} and is pronounced \textipa{[nEnE]}.

\section{Phonology}

\begin{tabular}{c | c | c | c | c | c | c | c}
  & Bilabial & Labiodental & Dental & Alveolar & Palatal & Velar & Glottal \\
\hline
Plosive & \textipa{p} & & & \textipa{t} & \textipa{c} & \textipa{k} & \\
\hline
Nasal & \textipa{m} & \textipa{M} & & \textipa{n} & & & \\
\hline
Fricative & & \textipa{f} & \textipa{T} & \textipa{s} & & & \textipa{h} \\
\hline
Approximant & & & & & \textipa{j} & & \\
\end{tabular}

\begin{tabular}{c | c | c | c }
\multirow{2}{2em}{} & \multicolumn{2}{| c |}{Front} & \multirow{2}{2em}{Back} \\
 & Unrounded & Rounded & \\
\hline
 Close & \textipa{i} & \textipa{y} & \textipa{u} \\
\end{tabular}

\section{Vocabulary}

\section{Grammar}

\section{Writing}




\end{document}
