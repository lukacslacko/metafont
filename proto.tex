
\documentclass[a4paper]{article}
\newfont{\foo}{proto}
\usepackage{graphicx}
\usepackage{textcomp}
\usepackage{tipa}
\usepackage{multirow}
\usepackage{float}

\title{Language}
\author{L\'aszl\'o Luk\'acs}

\begin{document}

\newcommand{\DIG}[1]{\reflectbox{\foo #1}}
\newcommand{\DIGA}{\DIG{A}}
\newcommand{\DIGB}{\DIG{B}}
\newcommand{\DIGC}{\DIG{C}}
\newcommand{\DIGD}{\DIG{D}}
\newcommand{\DIGE}{\DIG{E}}
\newcommand{\DIGF}{\DIG{F}}

\newcommand{\LET}[1]{\rotatebox[origin=c]{-90}{\reflectbox{\foo #1}}}
\newcommand{\VU}{\LET W}
\newcommand{\YOE}{\LET X}
\newcommand{\JEE}{\LET Z}
\newcommand{\NE}{\LET a}
\newcommand{\MOE}{\LET b}
\newcommand{\KI}{\LET c}

\newcommand{\SP}{\LET Y}

\newcommand{\DOWN}[1]{\rotatebox[origin=c]{-90}{#1}}
\newcommand{\UP}[1]{\rotatebox[origin=c]{-90}{\reflectbox{#1}}}

\newcommand{\Q}[1]{\textlangle #1\textrangle}

\maketitle

\tableofcontents

\DOWN{
\DIG0 \DIG1 \DIG2 \DIG3 \DIG4 \DIG5
}
\UP{
\DIG6 \DIG7 \DIG8 \DIG9 \DIGA\ \DIGB\
}
\DOWN{
\DIGC\ \DIGD\ \DIGE\ \DIGF\ \VU\ \YOE
}
\UP{
\VU\ \YOE\ \JEE\ {\foo .}\ \MOE\ \KI\
}
\DOWN{
\KI\ \SP\ \NE\ {\foo ,}\ \YOE\ \VU
}

The symbol for \Q{nene} is \DOWN{\NE\ \NE} and is pronounced \textipa{[nEnE]}.

\section{Phonology}

\begin{table}[H]
\begin{tabular}{c | c | c | c | c | c | c | c}
  & Bilabial & Labiodental & Dental & Alveolar & Palatal & Velar & Glottal \\
\hline
Plosive & \textipa{p} & & & \textipa{t} & \textipa{c} (y) & \textipa{k} & \\
\hline
Nasal & \textipa{m} & \textipa{M} (w) & & \textipa{n} & & & \\
\hline
Fricative & & \textipa{f} & \textipa{T} (v) & \textipa{s} & & & \textipa{h} \\
\hline
Approximant & & & & & \textipa{j} & & \\
\end{tabular}
\caption{Consonants}
\end{table}

\begin{table}[H]
\begin{tabular}{c | c | c | c }
\multirow{2}{2em}{} & \multicolumn{2}{| c |}{Front} & \multirow{2}{2em}{Back} \\
\cline{2-3}
 & Unrounded & Rounded & \\
\hline
 Close & \textipa{i} & \textipa{y} (\"u) & \textipa{u} \\
\hline
Close-mid & \textipa{e} (\^e) & \textipa{\o} (\"o) & \textipa{o} \\
\hline
Open-mid & \textipa{E} (e) & & \\
\end{tabular}
\caption{Vowels}
\end{table}

\subsection{Vowel harmony}

Words contain either close and open-mid (i, \"u, u, e) or close-mid and
open-mid (\^e, \"o, o, e) vowels.

\subsection{Stress}

In each word the first syllable is stressed.

\subsection{Syllables}

Each syllable has a CV shape.

\section{Vocabulary}

\section{Grammar}

\subsection{Word order}

\begin{itemize}
\item SVO: Subject, Verb, Object.
\item Noun - adjective
\item Posessor - posessee
\item Preposition - noun
\end{itemize}

\subsection{Plurals}

Plurals have the \Q{-m\^e, -mi} suffix. For example:
\begin{tabular}{c | c}
\emph{y\"one} (eye) & \emph{y\"onem\^e} (eyes) \\
\hline
\emph{fij\"une} (mouth) & \emph{fij\"unemi} (mouths)
\end{tabular}

\subsection{Moods}

\begin{table}[H]
\begin{tabular}{c | c}
Interrogative & \emph{n\"un\"u-, n\"on\"o-} \\
\hline
Negative & \emph{ne-} \\
\hline
Abilitative & \emph{joko-, juku-} \\
\hline
Negative & \emph{ne-}
\end{tabular}
\caption{Mood prefixes}
\end{table}

Examples:
\begin{table}[H]
\begin{tabular}{l | l}
\emph{ko y\"o} & I see \\
\hline
\emph{ko ney\"o} & I do not see \\
\hline
\emph{ko jokoy\"o} & I can see \\
\hline
\emph{ko nejokoy\"o} & I cannot see \\
\hline
\emph{ko nejokoney\"o} & I must see (lit.\ I cannot not see) \\
\hline
\emph{ko n\"on\"oy\"o} & Do I see? \\
\hline
\emph{ko n\"on\"oney\"o} & Do I not see?
\end{tabular}
\end{table}

\subsection{Imperfective aspect}

The imperfective aspect is marked between the moods prefixes and the verb by
\emph{yes\"o-, yes\"u}: 
\begin{tabular}{c | c}
\emph{ko nefij\"u} & I am not eating \\
\hline
\emph{ko neyes\"ufij\"u} & I have not eaten
\end{tabular} 

\subsection{Tense}

The tense is marked between the imperfective aspect and the verb by
\emph{pi-, p\^e} for the future and by \emph{vimu-, v\^emo-} for the past:
\emph{ko piy\"o} (I will see), \emph{ko v\^emoy\"o} (I saw).

\subsection{Passive}

Passive is marked by the \emph{-je} suffix: \emph{ko y\"oje} (I am seen).

\subsection{Examples}

\emph{ko nejokoyes\"ov\^emofij\"uje} -- I could not have been eaten.

\section{Writing}




\end{document}
